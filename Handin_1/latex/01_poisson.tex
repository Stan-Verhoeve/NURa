\section{Poisson distribution}
The Poisson probability distribution function, for integer $k$ and positive mean $\lambda$, is given by
\begin{equation}
	P_\lambda(k) = \frac{\lambda^k e^{-\lambda}}{k!}.
\end{equation}
To avoid under- and overflow, we do not evaluate the Poisson distribution as given. Instead, we transform to logarithmic space, such that
\begin{equation}
	\log\left(P_\lambda(k)\right) = \log\left(\frac{\lambda^k e^{-k}}{k!}\right)
\end{equation}
Using standard rules for logarithmic calculus, this can be written as
\begin{equation}
	\log\left(P_\lambda(k)\right) = k\log{\lambda} - k - \sum_k \log{k}.
\end{equation}
Important to note here is the limit of the sum. For $k\in[0,1]$, this sum should equal 1. As such, we can write the sum as the following set of cases:
\begin{equation}
	\sum_k \log{k} = \begin{cases}
				1 & 0 \leq k < 2\\
				\sum_{i=2}^k \log{i} & k \geq 2
			 \end{cases}.
\end{equation}
The above explanation is numerically implemented and tested \texttt{01\_poisson\_distribution.py}:

\lstinputlisting[style=pystyle,language=Python]{../01\_poisson\_distribution.py}

Output of the code is given in \texttt{OUT/01\_poisson\_distribution.txt}:

\lstinputlisting[style=txtstyle]{../OUT/01\_poisson\_distribution.txt}
